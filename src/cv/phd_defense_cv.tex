\documentclass[12pt,sans]{moderncv}

% moderncv themes
\moderncvstyle{classic}
\moderncvcolor{green}

\usepackage[a4paper, hmargin=1in, vmargin=1in]{geometry}
\usepackage[a-2u]{pdfx}
\usepackage[czech]{babel}
\usepackage[utf8]{inputenc}
\usepackage[T1]{fontenc}
\usepackage{lmodern}
\usepackage{textcomp}

% personal data
\name{\mbox{Jakub}}{\mbox{Klener}}
\title{Curriculum vit\ae{}}
\address{Eliášova 460/35}{160 00, Praha 6 -- Bubeneč}
\phone[mobile]{722\,411\,643}
\email{jakub.v.klener@gmail.com}
\homepage{fu.mff.cuni.cz/biomolecules/Klener\_Jakub/}


\begin{document}

\makecvtitle

\rmfamily
\hfill{}15. prosince 2021

\vspace{5mm}

\setlength{\parindent}{1.5em}

Pocházím ze severočeského města Litměřice, kde jsem v~roce 2006 odmaturoval na
Gymnáziu Josefa Jungmana.

Poté jsem nastoupil na bakalářský program Obecná fyzika na
Matematicko-fyzikální fakultě Univerzity Karlovy, který jsem úspěšně zakončil
v~roce 2009 obhájením bakalářské práce
\emph{Studium funkce prokaryotních homologů Nramp transportních proteinů}.

Po bakalářském studiu jsem pokračoval v~navazujícím magisterském studiu v~oboru
Biofyzika a~chemická fyzika, ze kterého mi byl v roce 2011 udělen magisterský
titul s~vyznamenáním a~má diplomová práce
\emph{Diagnostika neurodegenerativních chorob pomocí Ramanovy spektroskopie}
zvítězila v~soutěži Spektroskopické společnosti Jana Marca Marci za nejlepší
diplomovou práci v~oboru spektroskopie za rok 2011.
Výsledky diplomové práce byly také publikovány v článku
\emph{Instability of cerebrospinal fluid after delayed storage and repeated
freezing: a holistic study by drop coating deposition Raman spectroscopy}.

Na magisterské studium jsem navázal postgraduálním studiem na Oddělení
biomolekul Fyzikálního Ústavu Univerzity Karlovy pod vedením prof. RNDr. Josefa
Štěpánka, CSc. v oboru Biofyzika, chemická a makromolekulární fyzika.
Tématem mé dizertační práce je
\emph{Studium struktury a interakcí nukleových kyselin pomocí rezonančního
Ramanova rozptylu},
přičemž stěžejním cílem bylo zkonstruovat UV Ramanův spektrometr.
Na tomto spektrometru se následně vyzkoušela velká řada různých problematik
s různou měrou úspěšnosti,
	přes testování povrchů pro UV SERS,
	měření agregace tau proteinu s heparinem a různými inhibitory,
	měření TRPM proteinu,
	studium plodnosti ze spermatu,
	řasy amphidinium pěstované v prostředí s $^{15}$N,
	měření třikrát protonovaného adeninu v extrémě acidickém prostředí
až po tři studie shrnuté v dizertační práci zaměřené na studium
	RNA trojšroubovic,
	DNA vlásenek a
	DNA kvadruplexů,
z čehož 2 vyšly jako publikace v mezinárodních časopisech.
Některé výsledky také byly prezentovány na vědeckých konferencích.
V letech 2012 -- 2014 jsem byl řešitelem studentského grantu Grantové agentury
Univerzity Karlovy \emph{UV rezonanční Ramanův rozptyl nukleových kyselin}.

Kromě samotné UV Ramanovy spektroskopie jsem se také věnoval automatizaci
Ramanova spektrometru pro měření ve viditelné oblasti, který je používaný
na pracovišti Oddělení biomolekul, provádění a analýze molekulárně dynamických
simulací, výuce úvodu do programování v C++, vedl jsem jeden studentský projekt
a jednu maturitní práci na Gymnáziu Bohumila Hrabala v Numburce, která byla
úspěšně obhájena v roce 2017.

V roce 2017 jsem také nastoupil na částečný úvazek jako softwarový vývojář
tomografického software pro průmyslová CT.
V současné době pracuji na plný úvazek jako Lead Software Engineer ve firmě
Alteryx.
Jsem ženatý a mám čtyřletou dceru.


\end{document}
